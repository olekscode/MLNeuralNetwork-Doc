An artificial neural networks is one of the most developed and widely used algorithms of machine learning.
It is the mathematical model of brain's activity that is able to tackle both problems of classification and regression. Neural network can function as a model of supervised, unsupervised or reinforcement learning. 
  
\paragraph{Definition}
Simon Haykin \cite{Haykin-2005} offers the following definition:
\begin{quote}
  A neural network is a massively parallel distributed processor made up of simple processing units, which has a natural propensity for storing experiential knowledge and making it available for use. It resembles brain in two respects:
  \begin{enumerate}
    \item Knowledge is aquired by network from its environment through a learning process.
    \item Interneuron connection strengths, known as synaptic weights, are used to store the acquired knowledge.
  \end{enumerate}
\end{quote}

Since their invention in 50-s neural networks have been used to model human brain and approach the goal of creating human-like artificial intelligence. Nowadays it is more common to think of neural networks as of the statistical models that perform well on some extremely complicated tasks. For example, Hastie et. al. \cite{Hastie-et-al-2013} view neural networks as nonlinear statistical models, the two-stage regression or classification models. David MacKay \cite{MacKay-2003} sees them as parallel distributed computational systems consisting of many interacting simple elements. And Goodfellow et. al. (MIT) \cite{Goodfellow-et-al-2016} write the following

\begin{quote}
  Modern neural network research is guided by many mathematical and engineering disciplines, and the goal of neural networks is not to perfectly model the brain. It is best to think of feedforward networks as function approximation machines that are designed to achieve statistical generalization, occasionally drawing some insights from what we know about the brain, rather than as models of brain function.
\end{quote}
