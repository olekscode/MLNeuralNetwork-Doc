In recent years Machine Learning became one of the most promising and rapidly developing fields in Computer Science. It tackles the problems that classical programming and sometimes also humans can't handle. In this section I will give the short introduction to the field of Machine Learning.

In his book \textit{Information Theory, Inference, and Learning Algorithms} \cite{MacKay-2003} David MacKay writes:
  \begin{quote}
    Machine learning allows us to tackle tasks that are too difficult to solve with fixed programs written and designed by human beings. From a scientific and philosophical point of view, machine learning is interesting because developing our understanding of machine learning entails developing our understanding of the principles that underlie intelligence.
  \end{quote}
  
  \paragraph{Definition}
  % TODO: Cite something here
  The intuitive definition of machine learning was given by Arthur Samuel in 1959:
  
  \begin{quote}
    Machine Learning is a field of study that gives computers the ability to learn without being explicitly programmed.
  \end{quote}
  
  This definition is nice and easy to understand. Though, to work with machine learning as a scientific field (indeed, it is a field of computer science, strongly related to some mathematical fields, such as computational statistics and mathematical optimization), we need a more formal definition. One can be found in Tom Mitchell's book \textit{Machine Learning} (1997) \cite{Mitchell-1997}. This definition is widely known and often reffered to as a well-posed learning problem:
  
  \begin{quote}
  A computer program is said to learn from experience $ E $ with respect to some class of tasks $ T $ and performance measure $ P $, if its performance at tasks in $ T $, as measured by $ P $, improves with experience $ E $.
  \end{quote}
  
  \paragraph{Machine learning problems}
  By the way we measure performance $ P $ on task $ T $ machine learning tasks can be divided into three main classes:
  
  \begin{itemize}
    \item \textbf{Supervised Learning} - the agent receives the set of examples with labels ("right answers") to learn from
    \item \textbf{Unsupervised Learning} - no explicit feedback is provided. The agent should learn patterns in an unlabeled dataset
    \item \textbf{Reinforcement Learning} - the agent receives series of reinforcements - rewards or punishments (for example, winning or loosing the chess game)
  \end{itemize}
  
  Here are the formal definitions for the problems of supervised and unsupervised machine learning\footnote{Formal definitions of supervised and unsupervissed learning problems are inspired by \cite{Russell-Norvig-2010} and \cite{Raina-et-al-2009} respectively}.
  
  \paragraph{Task of supervised learning}
  Given a training set of $ m $ example input-output pairs
  \[ (x^{(1)}, y^{(1)}), (x^{(2)}, y^{(2)}), ..., (x^{(m)}, y^{(m)}) \]
  where each $ y^{(j)} $ is generated by an unknown function $ y^{(j)} = f(x^{(j)}) $
  
  \textbf{Goal:} discover the function $ h $ that approximates function $ f $.
  
  \paragraph{Task of unsupervised learning}
  Given a large unlabeled dataset of $ m $ input examples
  \[ x^{(1)}, x^{(2)}, ..., x^{(m)} \]
  where each $ x^{(i)} \in \mathbb{R}^{n} $.
  
  \textbf{Goal:} learn a model for the inputs and then apply it to a specific machine learning task.
  
  %TODO: Add task of reinforcement learning
  %TODO: Write about classification and regression
