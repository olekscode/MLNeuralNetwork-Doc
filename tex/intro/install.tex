To use my library you must first install Pharo. It is available for all major operating systems

\subsection{Getting Pharo}
Pharo is a modern, open-source, dynamically typed language supporting live coding and inspired by Smalltalk. The important principle behind Pharo is that it doesn’t just copy the past, but reinvents the essence behind Smalltalk. Pharo is not read-only, it integrates the changes made by community, daily\cite{PBE2}.

To download Pharo, visit the official website: \url{http://pharo.org/download}.

\subsection{Getting the code}
The code of this project can be acquired from Smalltalkhub using this Metacello script (\texttt{Do It} in a Playground of your Pharo image):

\begin{lstlisting}
Metacello new 
    repository: 'http://smalltalkhub.com/mc/Oleks/NeuralNetwork/main';
    configuration: 'MLNeuralNetwork';
    version: #development;
    load.
\end{lstlisting}

Or you can get it from the GitHub repository: \url{https://github.com/olekscode/MLNeuralNetwork}. In future this project will be moved to GitHub.

Parts of this project depend on the following packages. They will be automatically installed and updated by Metacello.
\begin{itemize}
  \item \textbf{PolyMath} - a library for numerical methods. It provides MATLAB-like vectors and matrices. It is simmilar to numpy library in Python.
  \item \textbf{Roassal} - a library for agile visualizations. The only part dependent on Roassal is MLVisualizer class. All data and metrics are provided in a type that is supported either by Pharo base, or by PolyMath. So if you wish to use something other visualization tool - you are free to do that. However, all examples in this [thesis] will be visualised with MLVisualizer which uses Roassal.
  \item \textbf{IdxReader} - a package for reading the data in idx format, designed by [Guilermo Polito]. The MLDataReader class uses it to read the MNIST dataset of handwritten digits. If you don't want to use this dataset - you can ignore this dependency. Everything else will work just fine.
\end{itemize}
