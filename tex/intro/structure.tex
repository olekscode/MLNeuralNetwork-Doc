\paragraph{First chapter} In this chapter will explain what is Pharo, how it can be installed, and how to acquire the code of this project. I will also talk about the ideas behind implementing neural networks, as well as the other machine learning algorithms, in an object-oriented manner, and give a brief overview of the work done before in this field.
\paragraph{Second chapter} Here I will give you the short introduction into machine learning and neural networks. I will talk about different kinds of learning, different network architectures, different ways of training them. Feel free to skip that chapter if you already have a good knowledge of the concepts in question. But do have a look at a section [?] where I talk about the confusing terminology and explain my reasoning behind naming the classes in this project.
\paragraph{Third chapter} Here I talk about my implementation of a single-layer perceptron as a collection of neurons. Each neuron in this representation is a separate object. This approach may not be the best one performance-wise, but it can be used for modeling purposes.
\paragraph{Fourth chapter} This chapter is dedicated to a multi-layer neural network. It is implemented as a collection of layers. Neurons are not the logical units anymore which allows us to perform some very expensive computations much faster using the vector-matrix operations implemented in the PolyMath package.
\paragraph{Fifth chapter} In this final chapter I demonstrate how my multi-layer neural network can be used for the classic task of classifying the handwritten digits from MNIST dataset. This chapter gives you the detailed explaination of how to use the neural networks package and how to extend it to fit your specific needs. So if you want to start using the package right ahead - jump straight to the fifth chapter.
