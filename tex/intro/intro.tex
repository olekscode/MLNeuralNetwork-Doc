The best things about object-oriented programming are flexibility, extendability and reusability. The field of neural networks is a very wide, complicated, and continuously growing. It requires different methods, state-of-the-art solutions, and various heuristics. Building a flexible neural networks framework and a modeling tool that would allow you to extend and modify it based on your specific problem, is a good task for OOP.

In this work I will describe my implementation of neural networks in Pharo in an object-oriented manner. One important think to note is that I do not claim to offer any new approach to building neural networks. This field has been well studied long before I got introduced to it. The main goal of this project is to bring neural networks to Smalltalk. To show how this beautiful language can be used for the problems of machine learning and data analysis. And to make my contribution into something that may later become a full-scale machine learning toolkit.

\section{Structure}
\paragraph{First chapter} In this chapter will explain what is Pharo, how it can be installed, and how to acquire the code of this project. I will also talk about the ideas behind implementing neural networks, as well as the other machine learning algorithms, in an object-oriented manner, and give a brief overview of the work done before in this field.
\paragraph{Second chapter} Here I will give you the short introduction into machine learning and neural networks. I will talk about different kinds of learning, different network architectures, different ways of training them. Feel free to skip that chapter if you already have a good knowledge of the concepts in question. But do have a look at a section [?] where I talk about the confusing terminology and explain my reasoning behind naming the classes in this project.
\paragraph{Third chapter} Here I talk about my implementation of a single-layer perceptron as a collection of neurons. Each neuron in this representation is a separate object. This approach may not be the best one performance-wise, but it can be used for modeling purposes.
\paragraph{Fourth chapter} This chapter is dedicated to a multi-layer neural network. It is implemented as a collection of layers. Neurons are not the logical units anymore which allows us to perform some very expensive computations much faster using the vector-matrix operations implemented in the PolyMath package.
\paragraph{Fifth chapter} In this final chapter I demonstrate how my multi-layer neural network can be used for the classic task of classifying the handwritten digits from MNIST dataset. This chapter gives you the detailed explaination of how to use the neural networks package and how to extend it to fit your specific needs. So if you want to start using the package right ahead - jump straight to the fifth chapter.


\section{Related work}
In 1995 Barry Ellingsen developed MANN (Modular Artificial Neural Network) system and described it in a technical report \textit{An Object-Oriented Approach to Neural Networks}\cite{Ellingsen-1995}. I took some inspiration from his work, but my implementation was also infuenced by modern packages, such as TensorFlow or scikit-learn.

There were several attempts of implementing neural networks in Pharo. The most notable one would be the work of Alexandre Bergel done in his package


\section{Installation}
To use my library you must first install Pharo. It is available for all major operating systems

\subsection{Getting Pharo}
Pharo is a modern, open-source, dynamically typed language supporting live coding and inspired by Smalltalk. The important principle behind Pharo is that it doesn’t just copy the past, but reinvents the essence behind Smalltalk. Pharo is not read-only, it integrates the changes made by community, daily\cite{PBE2}.

To download Pharo, visit the official website: \url{http://pharo.org/download}.

\subsection{Getting the code}
The code of this project can be acquired from Smalltalkhub using this Metacello script (\texttt{Do It} in a Playground of your Pharo image):

\begin{lstlisting}
Metacello new 
    repository: 'http://smalltalkhub.com/mc/Oleks/NeuralNetwork/main';
    configuration: 'MLNeuralNetwork';
    version: #development;
    load.
\end{lstlisting}

Or you can get it from the GitHub repository: \url{https://github.com/olekscode/MLNeuralNetwork}. In future this project will be moved to GitHub.

Parts of this project depend on the following packages. They will be automatically installed and updated by Metacello.
\begin{itemize}
  \item \textbf{PolyMath} - a library for numerical methods. It provides MATLAB-like vectors and matrices. It is simmilar to numpy library in Python.
  \item \textbf{Roassal} - a library for agile visualizations. The only part dependent on Roassal is MLVisualizer class. All data and metrics are provided in a type that is supported either by Pharo base, or by PolyMath. So if you wish to use something other visualization tool - you are free to do that. However, all examples in this [thesis] will be visualised with MLVisualizer which uses Roassal.
  \item \textbf{IdxReader} - a package for reading the data in idx format, designed by [Guilermo Polito]. The MLDataReader class uses it to read the MNIST dataset of handwritten digits. If you don't want to use this dataset - you can ignore this dependency. Everything else will work just fine.
\end{itemize}


\section{Example of usage}
To create an instance of MLNeuralNetwork class you must provide an inforamtion about the network architecture, defined by a simple one-dimensional array of integers $\#(s_{1} s_{2} ... s_{n})$. The size of an array $n$ represents the number of layers in the network, and each number $s_{i}$ is the number of neurons in that layer.

\begin{lstlisting}
neuralNet := MLNeuralNetwork new initialize: #(784 500 10).
\end{lstlisting}

This code will create a neural network with 784 input units, one hidden layer with 500 neurons, and an output layer with 10 neurons (this network can be used to classify the input data into 10 classes).


\section{How to contribute?}
Contributions are welcome and encouraged. The easiest way to contribute is to make a pull request to the GitHub repository of this project: \url{https://github.com/olekscode/MLNeuralNetwork}. If you have any questions or suggestions regarding this project, write me an email to \texttt{olk.zaytsev@gmail.com}.

If you want to implement some machine learning algorithms in Pharo - join us on the mailing list or on the \#polymath channel of Pharo's server on Discord.

